%%%
%%% Annual Cognitive Science Conference
%%% Sample LaTeX Paper -- Proceedings Format
%%%

% Original : Ashwin Ram (ashwin@cc.gatech.edu)       04/01/1994
% Modified : Johanna Moore (jmoore@cs.pitt.edu)      03/17/1995
% Modified : David Noelle (noelle@ucsd.edu)          03/15/1996
% Modified : Pat Langley (langley@cs.stanford.edu)   01/26/1997
% Latex2e corrections by Ramin Charles Nakisa        01/28/1997
% Modified : Tina Eliassi-Rad (eliassi@cs.wisc.edu)  01/31/1998
% Modified : Trisha Yannuzzi (trisha@ircs.upenn.edu) 12/28/1999
% Modified : Mary Ellen Foster (M.E.Foster@ed.ac.uk) 12/11/2000
% Modified : Ken Forbus                              01/23/2004
% Modified : Eli M. Silk (esilk@pitt.edu)            05/24/2005
% Modified : Niels Taatgen (taatgen@cmu.edu)         10/24/2006
% Modified : David Noelle (dnoelle@ucmerced.edu)     11/19/2014
% Modified : Roger Levy (rplevy@mit.edu)             12/31/2018
% Modified : Stephanie Denison                       11/29/2025
% Modified : Dae Houlihan (daeda@mit.edu)            12/01/2025


%%% Change "letterpaper" in the following line to "a4paper" if you must.

\documentclass[10pt,letterpaper]{article}

\usepackage{cogsci}

% \cogscifinalcopy %%% Uncomment this line for the final submission

%%% Bibliography %%%
\usepackage[
  style=apa,
  natbib=true,
  annotation=false,
]{biblatex}
\addbibresource{cogsci_bibliography_template.bib} %%% Specify the path to a BibLaTeX file
\setlength{\bibhang}{.125in}

\usepackage{float} %%% Roger Levy added this and changed figure/table placement to [H] for conformity to Word template, though floating tables and figures to top is still generally recommended!

% Sometimes it can be useful to turn off hyphenation for purposes such as spell checking of the resulting PDF.
% \usepackage[none]{hyphenat} %%% Uncomment to turn off hyphenation

\title{How to Make a Proceedings Paper Submission}

%%% Format authors using helper functions from authblk package %%%
\author[1]{\mbox{Author N. One (a1@uni.edu)}}
\author[2]{\mbox{Author Number Two}}
\affil[1]{Department of Hypothetical Sciences, University of Illustrations}
\affil[2]{Department of Example Studies, University of Demonstrations}

%%% Or, format authors manually %%%
% \author{
%   {\large\bfseries Author N. One (a1@uni.edu)$^1$ \& Author Number Two$^2$} \\
%   {\normalsize\normalfont
%     $^1$Department of Hypothetical Sciences, University of Illustrations \\
%     $^2$Department of Example Studies, University of Demonstrations
%   }
% }

\begin{document}

\maketitle

\begin{abstract}
Include no author information in the initial submission, to facilitate blind review. AI tools cannot
be listed as authors, and authors retain full responsibility for the accuracy, integrity, and
originality of all content in their manuscripts. This includes verifying factual claims, ensuring
proper attribution of ideas, and confirming that the work meets standards for academic integrity and
does not contain plagiarized content. See the Acknowledgments section of the template for AI use
declaration and acknowledgment. The abstract should be one paragraph, no more than 150~words,
indented 1/8~inch on both sides, in 9~point font with single spacing. The heading
``\textbf{Abstract}'' should be 10~point, bold, centered, with one line of space below it. This
one-paragraph abstract section is required only for standard proceedings papers. Following the
abstract should be a blank line, followed by the header ``\textbf{Keywords:}'' and a list of
descriptive keywords separated by semicolons, all in 9~point font, as shown below.

\textbf{Keywords:}
add your choice of indexing terms or keywords;
kindly use a semicolon;
between each term
\end{abstract}

\section{General Formatting Instructions}

The paper can be no longer than six pages plus an unlimited number of pages for references in the
\textbf{initial submission}. In the \textbf{final submission}, the text of the paper, including an
author line, must fit on six pages. An unlimited number of pages can be used for acknowledgments
and references.

The \textbf{title} should be in 14~point bold font, centered. The title should be formatted with
initial caps (the first letter of content words capitalized and the rest lower case). In the
\textbf{initial submission}, leave one space below the title and on the next line include the
phrase ``Anonymous CogSci submission'', centered, in 11~point bold font. In the \textbf{final
  submission}, leave one space below the title, then list author names (on one line, though if there
are many authors this will continue on subsequent lines) in 11~point bold font, and centered, with
superscript numerals that will correspond to author affiliation. The \textbf{corresponding
  author's} email address and no other email addresses should be placed in parentheses next to their
name in the author list. Starting on the next line, list authors' affiliations using the
corresponding superscript numeral and including only the department/unit and organization in
ordinary 10~point type, one affiliation per line.

The text of the paper should be formatted in two columns with an overall width of 7~inches
(17.8~cm) and length of 9.25~inches (23.5~cm), with 0.25~inches between the columns. Leave two line
spaces between the last author affiliation and the text of the paper; the text of the paper
(starting with the abstract) should begin no less than 2.75~inches below the top of the page. The
left margin should be 0.75~inches and the top margin should be 1~inch. \textbf{The right and bottom
  margins will depend on whether you use U.S. letter or A4 paper, so you must be sure to measure the
  width of the printed text}. Use 10~point Times Roman with 12~point vertical spacing, unless
otherwise specified.

Indent the first line of each paragraph by 1/8~inch (except for the first paragraph of a new
section). Do not add extra vertical space between paragraphs.

\section{First Level Headings}

First level headings should be in 12~point, initial caps, bold and centered. Leave one line space
above the heading and 1/4~line space below the heading.

\subsection{Second Level Headings}

Second level headings should be 11~point, initial caps, bold, and flush left. Leave one line space
above the heading and 1/4~line space below the heading.

\subsubsection{Third Level Headings}

Third level headings should be 10~point, initial caps, bold, and flush left. Leave one line space
above the heading, but no space after the heading.

\section{Formalities, Footnotes, and Floats}

Use standard APA citation format. Citations within the text should include the author's last name
and year. If the authors' names are included in the sentence, place only the year in parentheses,
as in \citet{DaphneEcho2022}, but otherwise place the entire reference in parentheses with the
authors and year separated by a comma \citep{DaphneEcho2022}. Use the ``et~al.'' construction for
works with three or more authors. List multiple references alphabetically and separate them by
semicolons \citep{August2007, DaphneEcho2022}.

\subsection{Footnotes}

Indicate footnotes with a number\footnote{Sample of the first footnote.} in the text. Place the
footnotes in 9~point font at the bottom of the column on which they appear. Precede the footnote
block with a horizontal rule.\footnote{Sample of the second footnote.}

\subsection{Tables}

Number tables consecutively. Place the table number and title (in 10~point) above the table with
one line space above the caption and one line space below it, as in Table~\ref{sample-table}. You
may float tables to the top or bottom of a column, and you may set wide tables across both columns.

\begin{table}[H]
  \begin{center}
    \caption{Sample table title.}
    \label{sample-table}
    \vskip 0.12in
    \begin{tabular}{ll}
      \hline
      Error type        & Example      \\
      \hline
      Take smaller      & 63 - 44 = 21 \\
      Always borrow~~~~ & 96 - 42 = 34 \\
      0 - N = N         & 70 - 47 = 37 \\
      0 - N = 0         & 70 - 47 = 30 \\
      \hline
    \end{tabular}
  \end{center}
\end{table}

\subsection{Figures}

All artwork must be very dark for purposes of reproduction and should not be hand drawn. Number
figures sequentially, placing the figure number and caption, in 10~point, after the figure with one
line space above the caption and one line space below it, as in Figure~\ref{sample-figure}. If
necessary, leave extra white space at the bottom of the page to avoid splitting the figure and
figure caption. You may float figures to the top or bottom of a column, and you may set wide
figures across both columns.

\begin{figure}[H]
  \begin{center}
    \fbox{CoGNiTiVe ScIeNcE}
  \end{center}
  \caption{This is a figure.}
  \label{sample-figure}
\end{figure}

\section{Acknowledgments}

In the \textbf{initial submission}, please only include acknowledgments of AI use and no other
acknowledgments to preserve anonymity. Regarding AI use: Authors may use AI tools when developing
their projects and preparing their manuscripts, but such use must be described, transparently and
in detail, in either the Methods or Acknowledgments section, as appropriate. Tools that are used to
improve spelling, grammar, and general editing are not included in the scope of these guidelines.
In the \textbf{final submission}, place acknowledgments (including human and AI contributions, and
funding information) in a section \textbf{at the end of the paper}.

\section{References Instructions}

Follow the APA Publication Manual for citation format, both within the text and in the reference
list, with the following exception: use the same format for unpublished references as for published
ones. Alphabetize references by the surnames of the authors, with single author entries preceding
multiple author entries. Order references by the same authors by the year of publication, with the
earliest first. Include DOIs if available.

Use a first level section heading, ``\textbf{References}'', as shown below. Use a hanging indent
style, with the first line of the reference flush against the left margin and subsequent lines
indented by 1/8~inch. Below are example references for a conference paper, journal article,
technical report, dissertation, book chapter, edited volume, and book, respectively.

\nocite{August2007}
\nocite{DaphneEcho2022}
\nocite{FitzgeraldGalli1985}
\nocite{Hakuole2001}
\nocite{Issa1963}
\nocite{Lobsang2023}
\nocite{MitanniNovember1972}

\printbibliography

\end{document}
